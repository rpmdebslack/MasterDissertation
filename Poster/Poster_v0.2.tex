\documentclass[final]{beamer}
\usepackage[orientation=portrait,size=a3,
            scale=0.4         % font scale factor
           ]{beamerposter}
           
\geometry{
  hmargin=1.75cm, % little modification of margins
}

%
\usepackage[utf8]{inputenc}
\usebackgroundtemplate{\includegraphics[width=\paperwidth]{Posterportrait-.png}}
\linespread{1.15}

\usetheme{sharelatex}

\title
[Stage de M2] % Conference
{ % Poster title
Optimal Configuration of Distribution Networks under\\[0.3ex]Technical Constraints based on Predictive Methods
}

\author{ % Authors
Bhargav Prasanna Swaminathan
}
\institute
[Grenoble INP] % General University
{
M2 - Master in Electrical Engineering for Smart Grids and Buildings\\
Research Domain: Electrical Power Systems - Optimization
}
\date{\today}

\setlength{\multicolsep}{6.0pt plus 2.0pt minus 1.5pt}% 50% of original values

\setlength{\topskip}{0pt}

\begin{document}
\begin{frame}[t]

\begin{columns}
\begin{column}{0.5\linewidth}

\subsection{Objectives}
The work here proposes an algorithm to perform a multi-objective, day-ahead scheduling / optimization for distribution networks based on predicted values of day-ahead DRES production and load consumption.%

\subsection{Methodology}
\begin{itemize}
\item Modular Algorithm with various replaceable components.
\item At $0$h, either start with original, or optimal configuration for the networks (best result shown in tables)
\item Tested on two networks (urban and rural), with economic criteria.
\item Original, and optimized network voltages for each hour shown in results: Minimum, Maximum, and Average voltage in p.u. 
\end{itemize}
\end{column}%

\begin{column}{0.5\linewidth}
% \vspace*{0.5cm}
\subsection{Algorithm Components}
\vspace*{-0.5cm}
\begin{figure}[!h]
% A component diagram will have to be included here.
\pgfdeclarelayer{background}
\pgfdeclarelayer{foreground}
\pgfsetlayers{background,main,foreground}

\tikzstyle{lcomp}=[draw, fill=blue!20, text width=5em, 
    text centered, minimum height=2.5em,text centered]

\tikzstyle{ann} = [above, text width=5em]

\tikzstyle{program} = [lcomp, text width=6em, fill=red!20, 
    minimum height=6em, minimum width=10em, rounded corners,text centered, drop shadow]

\tikzstyle{global} = [lcomp, text width=6em,
    fill=green!20, minimum height=6em, minimum width=8em, rounded corners,text centered, drop shadow]

\tikzstyle{reconf} = [lcomp, text width=8em, fill=gray!10, 
    minimum height=4em, minimum width=8em, rounded corners,text centered, drop shadow]

\tikzstyle{lf} = [lcomp, text width=8em, fill=yellow!20, 
    minimum height=4em, minimum width=8em, rounded corners,text centered, drop shadow]

\tikzstyle{local} = [draw, circle, text width=3em, fill=blue!20,
	minimum height=3em, minimum width=3em, text centered, circular drop shadow]

\tikzstyle{blank} = [node distance=1cm,minimum width=-1cm]

\tikzstyle{vecArrow} = [thick, decoration={markings,mark=at position 0 with {\arrowreversed[semithick]{open triangle 60}},mark=at position
   1 with {\arrow[semithick]{open triangle 60}}},
   double distance=1.4pt, shorten >= 5.5pt, shorten <= -5.5pt,
   preaction = {decorate},
   postaction = {draw,line width=1.4pt, white,shorten >= 4.5pt, shorten <= 4.5pt}]
\tikzstyle{innerWhite} = [semithick, white,line width=1.4pt, shorten >= 4.5pt, shorten <= -7pt]

\centering
\begin{tikzpicture}[thick,scale=0.7, every node/.style={transform shape}]
	% Main Program
    \node [program] (prog) {Main Program};

    % Reconfiguration
    \node [blank,right of=prog, node distance=1cm] (b1) {};
    \node [reconf,below of=prog, node distance=3.5cm] (rec) {Reconfiguration Algorithm};
    \node [reconf, right of=rec,node distance=5cm] (recdb) {Reconfiguration Database};
    \node [blank, left of=rec, node distance=2cm] (b5) {};

    % Load Flow
    \node [lf, above of=prog, node distance=3cm] (ld) {Load Flow};

    % Local Management
    \node [blank, below of=prog, node distance=3cm] (b2) {};
    \node [local, left of=prog, node distance=4.5cm] (lr) {LR};
    \node [local, left of=lr, node distance=2cm] (vvc) {VVC};
    \node [blank, left of=lr, node distance=1cm] (b3) {};
    \node [blank, right of=lr, node distance=1cm] (b4) {};
    \node [local, below of=b3, node distance=1.5cm] (oltc) {OLTC};
    \node [blank, above of=oltc, node distance=3cm] (lmg) {\large Local Management};

    % Global Management
    \node [global,right of=prog, node distance=5cm] (gm) {Global Management};

    \begin{pgfonlayer}{background}
        \path (vvc.west)+(-0.5,1.8) node (a) {};
        \path (oltc.south -| lr.east)+(+0.5,-0.2) node (b) {};
        \path[fill=blue!10,rounded corners, draw=black!50, dashed]
            (a) rectangle (b);

        \path (rec.north west)+(-0.5,0.5) node (c) {};
        \path (recdb.south east)+(+0.5,-0.5) node (d) {};
        \path[fill=gray!05,rounded corners, draw=black!50, dashed]
            (c) rectangle (d);
    \end{pgfonlayer}

    \draw[vecArrow] (prog.north)+(0,0.6) to (ld);
    \draw[innerWhite] (prog.north)+(0,0.6) to (ld);

    \draw[vecArrow] (lmg.north)+(0,0.6) |- (ld);
    \draw[innerWhite] (lmg.north)+(0,0.6) |- (ld);

    \draw[vecArrow] (gm.north)+(0,0.6) |- (ld);
    \draw[innerWhite] (gm.north)+(0,0.6) |- (ld);

	\draw[vecArrow] (b4)+(0.8,0) to (prog);
    \draw[innerWhite] (b4)+(0.8,0) to (prog);

    \draw[vecArrow] (prog.east)+(0.6,0) to (gm);
    \draw[innerWhite] (prog.east)+(0.6,0) to (gm);

	\draw[vecArrow] (rec.north)+(0,1.1) to (prog.south);
    \draw[innerWhite] (rec.north)+(0,1.1) to (prog.south);

    \draw[vecArrow] (recdb.north)+(0,1.1) to (gm.south);
    \draw[innerWhite] (recdb.north)+(0,1.1) to (gm.south);

    \draw[vecArrow] (gm.east)+(0.6,0) to node {} +(1.2,0) to node {} +(1.2,-5.3) to node {} +(-11.9,-5.3) to node {} +(-11.9,-2.4);
    \draw[innerWhite] (gm.east)+(0.6,0) to node {} +(1.2,0) to node {} +(1.2,-5.3) to node {} +(-11.9,-5.3) to node {} +(-11.9,-2.4);

\end{tikzpicture}
\end{figure}
% \setlength{\parskip}{0.3cm}
\end{column}
\end{columns}

\vspace*{0.3cm}
\noindent\makebox[\linewidth]{\rule{0.95\paperwidth}{0.4pt}}

\subsection{Results}

{\normalsize \textbf{PREDIS (Urban) Network}}

\begin{columns}
\begin{column}{0.8\linewidth}
\begin{figure}[!h]
\begin{minipage}[H]{.33\columnwidth}
	% \centering
	\newlength\figureheight
	\newlength\figurewidth
    \setlength\figureheight{4.5cm}
    \setlength\figurewidth{5.5cm}
	\input{Graphs/Results/PREDIS/OldLoad/Losses.tikz}
\end{minipage}%
\begin{minipage}[H]{.33\columnwidth}
	% \centering
	\vspace*{-0.3cm}
	\setlength\figureheight{4.5cm}
    \setlength\figurewidth{5.5cm}
	\input{Graphs/Results/PREDIS/OldLoad/VtgOpt.tikz}
\end{minipage}%
\begin{minipage}[H]{.33\columnwidth}
	% \centering
	\vspace*{-0.3cm}
	\setlength\figureheight{4.5cm}
    \setlength\figurewidth{5.5cm}
	\input{Graphs/Results/PREDIS/OldLoad/VtgOrg.tikz}
\end{minipage}%
\caption*{Figure - Losses and Voltage Profiles for the PREDIS (Urban) Network}
\label{fig:mandb}
\end{figure}
\end{column}
\begin{column}{0.25\linewidth}
\vspace*{-2.5cm}
\justifying
 \begin{table}[!hl]
% \centering
\begin{tabular}{ccc}
\textbf{Parameter} & \textbf{Original} & \textbf{Optimized}\\
& \textbf{Network} & \textbf{Network}\\
\hline
Losses(kWh) & 11720.8 & 2990.9\\
Money Spent(\euro) & 41057.7 & 14096.3 \\
Violations & 79 & 12\\
\hline
\end{tabular}
\end{table}
\ \\
\begin{itemize}
\item Significant loss reduction
\item Significant violation reduction
\end{itemize}
\end{column}
\end{columns}
\vspace*{0.5cm}
{\normalsize \textbf{Rural Network}}

\begin{columns}
\begin{column}{0.8\linewidth}
\begin{figure}[!h]
\begin{minipage}[H]{.33\columnwidth}
	% \centering
    \setlength\figureheight{4.5cm}
    \setlength\figurewidth{5.5cm}
	\input{Graphs/Results/Das/OldLoad/LossesBoth.tikz}
\end{minipage}%
\begin{minipage}[H]{.33\columnwidth}
	% \centering
	\vspace*{-0.3cm}
	\setlength\figureheight{4.5cm}
    \setlength\figurewidth{5.5cm}
	\input{Graphs/Results/Das/OldLoad/VtgOpt.tikz}
\end{minipage}%
\begin{minipage}[H]{.33\columnwidth}
	% \centering
	\vspace*{-0.3cm}
	\setlength\figureheight{4.5cm}
    \setlength\figurewidth{5.5cm}
	\input{Graphs/Results/Das/OldLoad/VtgOrg.tikz}
\end{minipage}%
\caption*{Figure - Losses and Voltage Profiles for the Rural Network}
\label{fig:mandb}
\end{figure}
\end{column}
\begin{column}{0.25\linewidth}
 \vspace*{-2.5cm}
\justifying
 \begin{table}[!hl]
% \centering
\begin{tabular}{ccc}
\textbf{Parameter} & \textbf{Original} & \textbf{Optimized}\\
& \textbf{Network} & \textbf{Network}\\
\hline
Losses(kWh) & 999.8 & 900.5\\
Money Spent(\euro) & 17132.9 & 5245.7 \\
Violations & 34 & 3\\
\hline
\end{tabular}
\end{table}
\ \\
\begin{itemize}
\item Significant violation reduction
\item Noticeable difference between the two approaches
\end{itemize}
\end{column}
\end{columns}

\noindent\makebox[\linewidth]{\rule{0.95\paperwidth}{0.4pt}}

\begin{columns}
\begin{column}{0.5\linewidth}
\subsection{Conclusions}
There is a significant improvement in the condition of the two networks, based on the predicted values of DRES and loads.
\setlength{\parskip}{0.3cm}

Most of the current literature: focuses on optimization considering only a ``snapshot'' (static loads and DRES production). Here, the conditions vary.
\setlength{\parskip}{0.3cm}

This work improves over the rest which consider varying conditions.
\end{column}

\begin{column}{0.5\linewidth}
\subsection{Future Work}
Future developments have been envisaged, as a part of work leading to a PhD:
\setlength{\parskip}{0.3cm}

\begin{itemize}
\item A novel state-of-the-art reconfiguration function.

\item A multi-objective constrained optimization function for deciding between various methods for violation management.

\item Development of an economic model that considers all the real-world factors.

\item Development of a day-ahead market-based purchase scenario for flexibilities.
\end{itemize}
% \subsection{References}
% \begin{thebibliography}{99}
% \bibitem{ref1} J.~Doe, Article name, \textit{Phys. Rev. Lett.}
% \bibitem{ref2} J.~Doe, J. Smith, Other article name, \textit{Phys. Rev. Lett.}
% \bibitem{web} \url{http://www.google.pl}
% \end{thebibliography}
\end{column}
\end{columns}
\end{frame}
\end{document}

